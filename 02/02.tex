\documentclass[11pt, oneside, BCOR0mm, DIV9, headinclude]{scrartcl}
\usepackage{scrpage2}
\usepackage{lmodern}
\usepackage[kerning=true, spacing=true]{microtype}
\usepackage[T1]{fontenc}
\usepackage[utf8]{inputenc}
\usepackage[ngerman]{babel}
\usepackage{amsmath}
\usepackage{mathtools}
\usepackage{amsthm}
\usepackage{amssymb}
\usepackage{bm}
\usepackage{graphicx}
\usepackage{pdfpages}
\usepackage{caption}
\usepackage{subcaption}
%\usepackage{fullpage}
%\usepackage{array}

\def\myTitle{Webengineering}
\def\myAuthor{B.~Strobel, K.~Krammer, L.~Hnatek, S.~Fischer}  % <----- HIER EURE NAMEN EINTRAGEN
\def\mySheetNo{Blatt 02}                                      % <----- AKT\"{u}LLE BLATTNUMMER EINTRAGEN
\def\myAbgabeDate{}

\usepackage[pdftex,colorlinks=true,
                      pdfstartview=FitV,
                      linkcolor=blue,
                      citecolor=blue,
                      urlcolor=blue,
                      breaklinks=true, 
                      plainpages=false,
                      pdfpagelabels
          ]{hyperref}
          \pdfinfo{
            /Title      (\myTitle)
            /Author     (\myAuthor)
} %%F\"{u}r Bookmarks im PDF-Dokument
\usepackage{url} 
\usepackage{color}
\usepackage{listings}

%%%%%%%%%%%%%%%%%%%%%%%%%
%Headereinstellungen    %
%%%%%%%%%%%%%%%%%%%%%%%%%

%\renewcommand{\chaptermark}[1]{ \markboth{#1}{}}
%Definiere hier einige Variablen die die Headings dann immer enthalten wenn man
% das Kommando auch einfuegt
\newcommand{\iHeadText}{\scriptsize{\myAuthor}}%{\leftmark} %links / inner
\newcommand{\iHeadTextO}{\scriptsize{\myAuthor}} %Bei pagestyle plain
\newcommand{\cHeadText}{} %mitte
\newcommand{\cHeadTextO}{}
\newcommand{\oHeadText}{\scriptsize\mySheetNo} %rechts /outer
\newcommand{\oHeadTextO}{\scriptsize\mySheetNo}
\newcommand{\iFootText}{\scriptsize  } %links
\newcommand{\iFootTextO}{\scriptsize  }
\newcommand{\cFootText}{} %mitte
\newcommand{\cFootTextO}{}
\newcommand{\oFootText}{\pagemark}
\newcommand{\oFootTextO}{\pagemark}
%Kommando zum Einsetzen des Headers
\newcommand{\setStandardHeader}{\ihead[\iHeadTextO]{\iHeadText} \chead[\cHeadTextO]{\cHeadText}
\ohead[\oHeadTextO]{\oHeadText} \ifoot[\iFootTextO]{\iFootText} 
\cfoot[\cFootTextO]{\cFootText} \ofoot[\oFootTextO]{\oFootText}
\automark[section]{chapter}} % %automark wichtig damit right und leftmark
%gesetzt werden

\pagestyle{scrheadings}
\clearscrheadfoot %alle bisherigen Einstellungen loeschen (eher optional)

%%%%%%%%%%%%%%%%%%%%%%%%%%%%%
%Headereinstellungen Ende   %
%%%%%%%%%%%%%%%%%%%%%%%%%%%%%

\setlength{\parskip}{0.4em}

%___________________________________________________________________________________________%
\begin{document}
\lstset{language=Matlab,numbers=left,captionpos=b}
\setStandardHeader

\begin{center}

\hrule\vspace{5pt}
\large\myAuthor\\
\normalsize Webengineering - \mySheetNo\\
WiSe 2013/14, Inst. f. Medieninformatik, Universität Ulm
\vspace{5pt}
\hrule
\end{center}


%% Blatt beginnt hier

\section*{02.~Aufgabe: HTTP}

\begin{enumerate}
    \item 	Eine idempotente HTTP-Methode kann mehrere male aufgerufen werden, ohne dass sich
		das Ergebnis "andert.

		Mit POST wird etwas erstellt.
		Durch die POST-Methode wird gefordert, dass der Server die mitgeschickte Repr"asentaton 
		akzeptiert, welche durch die Zielressource verarbeitet werden soll.

		Mit PUT wird etwas erstellt oder geupdated.
		Durch die PUT-Methode wird gefordert, dass der Zustand der Zielressource erzeugt oder 
		ersetzt wird durch den definierten Zustand der Repr"asentation, welche im Request 
		mitgeschickt wird.


		Beispiele:

		``POST/buch'' wird ein neues Buch kreiieren und
		der Server antwortet mit einer neue URL, die das Buch identifiziert.
		Also beispielsweise ``/buch/2''.

		``PUT/buecher/2'' wird entweder das neue Buch mit der ID 2 erstellen, oder das 
		bereits existierende Buch mit der ID 2 ersetzen.
		\begin{verbatim}http://tools.ietf.org/html/draft-ietf-httpbis-p2-semantics-21#section-5.3.4 \end{verbatim}


    \item       Methoden die ``safe'' sind, werden nie die Repr"asentation einer Ressource
	        manipulieren (Z.B. HEAD, OPTIONS und GET).
		\begin{verbatim}http://restcookbook.com/HTTP%20Methods/idempotency/\end{verbatim}

    \item       Der Zweck einer persistenten HTTP-Verbindung ist, die gleiche TCP-Verbindung zu
		nutzen um mehrere Requests/Responses zu senden/erhalten, statt immer wieder die 
		Verbindug zum Server neu aufzubauen. Der Client kann mehrere Requests senden, ohne 
		zuerst auf jeden Response warten zu m"ussen (effizienter in k"urzerer Zeit).
		
		Eine persistente Verbindung muss nicht aufgebaut werden, wenn sowieso nur ein
		Request/Response erfolgen soll.
		\begin{verbatim}http://tools.ietf.org/html/draft-ietf-httpbis-p2-semantics-21#section-5.3.4 \end{verbatim}

	\item   Die angegebene Seite \begin{verbatim}http://mobile3.informatik.uni-ulm.de/we01/cat.jpg\end{verbatim}
		ist eigentlich eine HTML Datei. Der Browser erkennt das und zeigt sie an.

		Da "uber CSS das Hintergrundbild f"ur den Body angegeben wurde, kann auch dieses angezeigt
		werden. Unabh"angig davon, dass auch der Link darauf mit HTML falsch referenziert wurde
		und eigentlich ein JPEG ist. Der Browser bestimmt vor dem Anzeigen des Elementes den Typen 
		selber, weshalb zuletzt das Katzenbild erscheint.
		\begin{verbatim}http://www.w3schools.com/cssref/pr_background-image.asp\end{verbatim}

\end{enumerate}


\end{document}

